%-----------------------------------------------------------------------------
%
%               Template for sigplanconf LaTeX Class
%
% Name:         sigplanconf-template.tex
%
% Purpose:      A template for sigplanconf.cls, which is a LaTeX 2e class
%               file for SIGPLAN conference proceedings.
%
% Guide:        Refer to "Author's Guide to the ACM SIGPLAN Class,"
%               sigplanconf-guide.pdf
%
% Author:       Paul C. Anagnostopoulos
%               Windfall Software
%               978 371-2316
%               paul@windfall.com
%
% Created:      15 February 2005
%
%-----------------------------------------------------------------------------


\documentclass{sigplanconf}

% The following \documentclass options may be useful:

% preprint      Remove this option only once the paper is in final form.
% 10pt          To set in 10-point type instead of 9-point.
% 11pt          To set in 11-point type instead of 9-point.
% authoryear    To obtain author/year citation style instead of numeric.

\usepackage{amsmath}


\begin{document}

\special{papersize=8.5in,11in}
\setlength{\pdfpageheight}{\paperheight}
\setlength{\pdfpagewidth}{\paperwidth}

\conferenceinfo{CONF 'yy}{Month d--d, 20yy, City, ST, Country} 
\copyrightyear{20yy} 
\copyrightdata{978-1-nnnn-nnnn-n/yy/mm} 
\doi{nnnnnnn.nnnnnnn}

% Uncomment one of the following two, if you are not going for the 
% traditional copyright transfer agreement.

%\exclusivelicense                % ACM gets exclusive license to publish, 
                                  % you retain copyright

%\permissiontopublish             % ACM gets nonexclusive license to publish
                                  % (paid open-access papers, 
                                  % short abstracts)

\titlebanner{banner above paper title}        % These are ignored unless
\preprintfooter{short description of paper}   % 'preprint' option specified.

\title{A Step Towards Formalized Forensics}
\subtitle{Subtitle Text, if any}

\authorinfo{Name1}
           {Affiliation1}
           {Email1}
\authorinfo{Name2\and Name3}
           {Affiliation2/3}
           {Email2/3}

\maketitle

\begin{abstract}
This is the text of the abstract.
\end{abstract}

\category{CR-number}{subcategory}{third-level}

% general terms are not compulsory anymore, 
% you may leave them out
\terms
term1, term2

\keywords
keyword1, keyword2

\section{Introduction}

There is no single definition of essential factors with regard to digital
forensics. Further, the de facto standards are within specific tools. What we
want is a formalization of what it means for a file to be malicious, deleted,
altered, etc. To do that, we use the Coq programming language and proof
environment to formalize a handful of statements along with some
hand-generated proofs for existing file images.

\section{JPEG File}

We start by considering a relatively straight forward request. I'd like to
show that a given file is a JPEG. How can we formalize that notion? We could
check the file extension, but that is not particularly reliable. Instead, we
will look at the JPEG spec (much easier said than done) and note that JPEG
files begin with a particular byte code and end with another. We will
therefore define the proof that a given file is a JPEG as the first type bytes
equal SOMECODE and the last two are SOMEOTHER. We could imagine other file
types being similarly defined.

Note that, while we were given an entire file, we only cared about the initial
and final bytes. As our definition does not involve any other bytes of the
image, these bytes could be all zeros, all ones, present, damaged, or
otherwise. The definition doesn't care, and this is a trait we will use in the
future to limit the size of our proofs.

\section{Delete INode}

Now, let's consider a more complicated example, that of testing whether a
particular inode index is marked as ``deleted'' in the file system. Note that
there are many possible definitions of ``deleted'' we may choose from. The
ext2 file system has redundant mechanisms to define if a file has been
deleted. We chose one, the allocation bit map. Ultimately, whether or not a
given inode index is marked as deleted involves finding the ``group block''
associated with that index, looking up the bit associated with the index in
the allocation table, and checking whether it is zero (for deleted) or one
(for allocated).

Of course, there is a lot we glossed over. What is a group block? How do we
find the right one? How do we know that the inode index provided is valid? To
answer these questions, we created several additional data structures and
evidence functions.

\section{Future Work}

automatically generate the proof based on disk images; additional types of
evidence; abstractions of file systems


\appendix
\section{Appendix Title}

This is the text of the appendix, if you need one.

\acks

Acknowledgments, if needed.

% We recommend abbrvnat bibliography style.

\bibliographystyle{abbrvnat}

% The bibliography should be embedded for final submission.

\begin{thebibliography}{}
\softraggedright

\bibitem[Smith et~al.(2009)Smith, Jones]{smith02}
P. Q. Smith, and X. Y. Jones. ...reference text...

\end{thebibliography}


\end{document}

%                       Revision History
%                       -------- -------
%  Date         Person  Ver.    Change
%  ----         ------  ----    ------

%  2013.06.29   TU      0.1--4  comments on permission/copyright notices

